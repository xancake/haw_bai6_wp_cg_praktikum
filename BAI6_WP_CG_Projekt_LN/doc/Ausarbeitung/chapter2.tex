\chapter{Partikelsysteme} \label{particlesystems}
\textmd{\textquotedblleft A particle system is a collection of many many minute particles that together represent a fuzzy object. Over a period of time, particles are generated into a system, move and change from within the system, and die from the system.\textquotedblright\quad\cite[S. 92]{reeves1983}}

\textmd{\\Partikelsysteme bestehen also aus einer Vielzahl von einzelnen Partikeln die durch das Partikelsystem verwaltet werden. Ein Partikel zeichnet sich haupts�chlich dadurch aus, dass er eine beschr�nkte Lebensdauer hat. Abh�ngig von der Lebenszeit k�nnen dann weitere Eigenschaften modelliert werden, wie zum Beispiel Geschwindigkeit, Beschleunigung, Masse oder Farbe.}

\textmd{\\�ber die Lebenszeit eines Partikelsystems werden st�ndig Partikel hinzugef�gt und entfernt. Je nach Art und Lebensdauer des Partikelsystems k�nnen dadurch sehr viele Partikel erzeugt werden, was bei der Implementation ber�cksichtigt werden muss. Au�erdem muss zwischen Partikelsystemen unterschieden werden, die dauerhaft leben und jenen die ein begrenztes Dasein fristen.}

\textmd{\\Partikelsysteme sind sehr vielseitig und k�nnen unterschiedlichste Ph�nomene modellieren. Pixar verwenden sie in ihren Filmen zum Beispiel f�r Staub, Rauch, Feuer und Wasser, vgl. \cite{pixar}. All diese Ph�nomene sind in ihrer Natur sehr unterschiedlich, doch sie unterliegen alle den physikalischen Gesetzen. Durch die Einhaltung der physikalischen Gesetze bei der Implementation k�nnen Partikelsysteme sehr realistisch simuliert werden. }
