\documentclass[
	draft=false,
	paper=a4,
	twoside=false,
	fontsize=11pt,
	headsepline,
	BCOR10mm,
	DIV11
]{scrbook}
\usepackage[ngerman,english]{babel}
%% see http://www.tex.ac.uk/cgi-bin/texfaq2html?label=uselmfonts
\usepackage[T1]{fontenc}
%\usepackage[utf8]{inputenc}
\usepackage[latin1]{inputenc}
\usepackage{ngerman}
\usepackage{lmodern}
\usepackage{libertine}
\usepackage{pifont}
\usepackage{microtype}
\usepackage{textcomp}
\usepackage[german,refpage]{nomencl}
\usepackage{setspace}
\usepackage{makeidx}
\usepackage{listings}
\usepackage{natbib}
\usepackage[ngerman,colorlinks=true]{hyperref}
\usepackage{soul}
\usepackage{hawstyle}
%%\usepackage[texencoding=utf8, backend=biber, style=din1505, block=none, mincrossrefs=9999]{biblatex}

%% define some colors
\colorlet{BackgroundColor}{gray!20}
\colorlet{KeywordColor}{blue}
\colorlet{CommentColor}{black!60}
%% for tables
\colorlet{HeadColor}{gray!60}
\colorlet{Color1}{blue!10}
\colorlet{Color2}{white}

%% configure colors
\HAWifprinter{
	\colorlet{BackgroundColor}{gray!20}
	\colorlet{KeywordColor}{black}
	\colorlet{CommentColor}{gray}
	% for tables
	\colorlet{HeadColor}{gray!60}
	\colorlet{Color1}{gray!40}
	\colorlet{Color2}{white}
}{}
\lstset{
	numbers=left,
	numberstyle=\tiny,
	stepnumber=1,
	numbersep=5pt,
	basicstyle=\ttfamily\small,
	keywordstyle=\color{KeywordColor}\bfseries,
	identifierstyle=\color{black},
	commentstyle=\color{CommentColor},
	backgroundcolor=\color{BackgroundColor},
	captionpos=b,
	fontadjust=true
}
\lstset{
	escapeinside={(*@}{@*)}, % used to enter latex code inside listings
	morekeywords={uint32_t, int32_t}
}
\ifpdfoutput{
	\hypersetup{bookmarksopen=false,bookmarksnumbered,linktocpage}
}{}

%% more fancy C++
\DeclareRobustCommand{\cxx}{C\raisebox{0.25ex}{{\scriptsize +\kern-0.25ex +}}}

\clubpenalty=10000
\widowpenalty=10000
\displaywidowpenalty=10000

% unknown hyphenations
\hyphenation{
}

%% recalculate text area
\typearea[current]{last}

\makeindex
\makenomenclature

\begin{document}
	\selectlanguage{ngerman}
	
	%%%%%
	%% customize (see readme.pdf for supported values)
	\HAWThesisProperties{
		Author={Lars Nielsen},
		Title={Partikelsysteme},
		EnglishTitle={Particle Systems},
		ThesisType={Projektarbeit},
		ExaminationType={Projektarbeit},
		DegreeProgramme={Bachelor of Science Angewandte Informatik},
		ThesisExperts={Prof. Dr. Jenke},
		ReleaseDate={16. M�rz 2017}
	}
	
	\frontmatter
	\maketitle
	\onehalfspacing
	
	%% note: this is one command on multiple lines
	\HAWAbstractPage
	%% German abstract
	{Partikelsysteme, Partikel}%
	{Dieses Dokument fasst die Erkenntnisse und Techniken zusammen, die zur Entwicklung eines Partikelsystems im Rahmen der Projektarbeit f�r das Wahlpflichtfach Computergraphik angewendet wurden.}
	%% English abstract
	{particle systems, particle}%
	{This document summarizes the findings and techniques that were used for developing a particle system within the scope of the project realized for Computergraphik.}
	
	\newpage
	\singlespacing
	
	\tableofcontents
	\newpage
	%% enable if these lists should be shown on their own page
	%%\listoftables
	%%\listoffigures
	%%\lstlistoflistings
	
	%% main
	\mainmatter
	\onehalfspacing
	
	\chapter{Einleitung}
\textmd{}

See also \cite{gamasutra2000}.

\section{Aufgabenstellung}
\textmd{Diese Projektarbeit befasst sich mit der Erstellung eines Partikelsystems.}

\section{Struktur der Arbeit}
\textmd{}

	\chapter{Partikelsysteme} \label{particlesystems}
\textmd{\textquotedblleft A particle system is a collection of many many minute particles that together represent a fuzzy object. Over a period of time, particles are generated into a system, move and change from within the system, and die from the system.\textquotedblright\quad\cite[S. 92]{reeves1983}}

\textmd{\\Partikelsysteme bestehen also aus einer Vielzahl von einzelnen Partikeln die durch das Partikelsystem verwaltet werden. Ein Partikel zeichnet sich haupts�chlich dadurch aus, dass er eine beschr�nkte Lebensdauer hat. Abh�ngig von der Lebenszeit k�nnen dann weitere Eigenschaften modelliert werden, wie zum Beispiel Geschwindigkeit, Beschleunigung, Masse oder Farbe.}

\textmd{\\�ber die Lebenszeit eines Partikelsystems werden st�ndig Partikel hinzugef�gt und entfernt. Je nach Art und Lebensdauer des Partikelsystems k�nnen dadurch sehr viele Partikel erzeugt werden, was bei der Implementation ber�cksichtigt werden muss. Au�erdem muss zwischen Partikelsystemen unterschieden werden, die dauerhaft leben und jenen die ein begrenztes Dasein fristen.}

\textmd{\\Partikelsysteme sind sehr vielseitig und k�nnen unterschiedlichste Ph�nomene modellieren. Pixar verwenden sie in ihren Filmen zum Beispiel f�r Staub, Rauch, Feuer und Wasser, vgl. \cite{pixar}. All diese Ph�nomene sind in ihrer Natur sehr unterschiedlich, doch sie unterliegen alle den physikalischen Gesetzen. Durch die Einhaltung der physikalischen Gesetze bei der Implementation k�nnen Partikelsysteme sehr realistisch simuliert werden. }

	\chapter{Implementation}
\textmd{In diesem Kapitel wird die Implementation des Partikelsystems aus dem Projekt erl�utert. Dabei wird auf die Techniken eingegangen, die zur Realisierung des Partikelsystems eingesetzt wurden.}

\section{Datenstrukturen}
\textmd{}
\textmd{TODO: Particle, Particle.Builder, ParticleSystem, ParticleSystemManager erkl�ren}

\section{Anziehen und Absto�en (?)}
\textmd{}
\textmd{TODO: Attraktoren und Repeller}
\cite{natureofcode2012} \cite{khan}

\section{Lebenszyklusmanagement}
\textmd{}
\textmd{TODO: ParticleSystemManager, Particle-Pool erkl�ren }

\section{Darstellung}
\textmd{Das Projekt verwendet den aus dem Praktikum bekannten Szenengraphen. Die Einbindung eines Partikelsystems in den Szenengraphen wird mit einem ParticleSystemNode realisiert. Dieser erh�lt bei seiner Erzeugung das Partikelsystem das er darstellen soll. Ab der Einbindung in den Szenengraphen ist der ParticleSystemNode f�r das Aktualisieren und Zeichnen des Partikelsystems zust�ndig. Zum Zeichnen wird ein VertexBufferObject der einzelnen Partikel erzeugt und OpenGL �bergeben. Zur fl�ssigen Darstellung des Partikelsystems geschieht das in jedem Renderzyklus. Um die H�ufigkeit des Renderzyklus festzulegen, kann �ber die Klasse ParticleSystemShowcaseScene eine gew�nschte Frames-Per-Second-Rate angegeben werden.}
\textmd{Die konkrete Darstellung der Partikel wird als farbige Punkte realisiert.}

\section{Transparenz}
\textmd{Standardm��ig werden die Partikel ohne eine bestimmte Sortierung gezeichnet. Das sorgt in OpenGL daf�r, dass Transparenz nicht richtig dargestellt werden kann. Wenn ein transparentes Objekt das n�her zur Kamera ist zuerst gezeichnet wird und danach eines das weiter hinten ist, so werden die Pixel des ersten Objekts nicht mehr angepasst. F�r ein Partikelsystem mit Transparenz sieht das aus wie in Abbildung \ref{transparency-no-btf}. In der Abbildung k�nnen dunkle Partikel beobachtet werden, die transparent sein sollten.}
\begin{figure}[h]
	\begin{center}
		\includegraphics[width=25em]{img/transparency_no_back-to-front_ordering.png}
		\caption{Partikelsystem ohne Back-to-Front-Sortierung}
		\label{transparency-no-btf}
	\end{center}
\end{figure}
\textmd{\\Um die Transparenz korrekt darzustellen wurde eine Back-to-Front-Sortierung unter Zuhilfenahme der Binary-Space-Partition berechnet. Bei der Binary-Space-Partition werden die Partikel r�umlich durch Hyperebenen in einer Baumstruktur getrennt \cite[Folie 26]{jenke2016}. Jeder Partikel befindet sich dann entweder vor oder hinter einer Hyperebene, bei der er einsortiert wurde. Betrachtet man nun einen Sichtpunkt, so kann anhand der Hyperebenen die Back-to-Front-Sortierung ermittelt werden \cite[Folie 32]{jenke2016}.}
\textmd{\\Die Back-to-Front-Sortierung wiederum ist wichtig um die Partikel in der richtigen Reihenfolge zu zeichnen, sodass OpenGL die Transparenz korrekt darstellt. In der Darstellung des Beispiels aus Abbildung \ref{transparency-no-btf} mit Back-to-Front-Sortierung (siehe Abbildung \ref{transparency-with-btf}) fallen nun keine dunklen Partikel mehr auf. }
\begin{figure}[h]
	\begin{center}
		\includegraphics[width=25em]{img/transparency_with_back-to-front_ordering.png}
		\caption{Partikelsystem mit Back-to-Front-Sortierung}
		\label{transparency-with-btf}
	\end{center}
\end{figure}

	\chapter{Zusammenfassung und Erkenntnisse} \label{summary}
\textmd{In diesem Kapitel werden Erkenntnisse widergespiegelt, die w�hrend der Implementation des Partikelsystems in dem Projekt aufgetreten sind.}

\section{Performance in der Update-Methode}
\textmd{Die Update-Methoden von Partikeln und Partikelsystemen werden mindestens 30 mal pro Sekunde aufgerufen um einen fl�ssigen Bildlauf zu garantieren. Dazu kommt, dass es pro Partikelsystem tausende von Partikeln geben kann. Deshalb sollten in den Update-Methoden so wenig Objekte wie m�glich erzeugt werden. Insbesondere immutable Operationen bzw. Datentypen sind hier sch�dlich, da sie neue Objekte erzeugen. Das bietet gro�es Potential f�r ein Speicherleck. In der Update-Methode der Partikel z.B. wurden urspr�nglich bei �nderung der Position und Geschwindigkeit st�ndig neue Vektoren erzeugt. Durch umstellen auf mutable Operationen konnte bei einem Partikelsystem mit 10.000 Partikeln der Speicherbedarf von 660MB auf 330MB reduziert werden.}

\section{Eigenschaften von Partikeln an Partikelsystemen �ndern}
\textmd{Sollen Eigenschaften von allen Partikeln eines Partikelsystems auf einmal ge�ndert werden, wie das zum Beispiel beim Anwenden von Kr�ften der Fall ist, muss darauf R�cksicht genommen, dass die update-Methode ggf. von einem anderen Thread ausgef�hrt wird. Bei einer Implementation die z.B. in der applyForce-Methode den Partikeln direkt durch eine Schleife eine Kraft zuweisen m�chte kann es passieren, dass sich die Sammlung der lebendigen Partikel konkurrierend �ndert, was zu einer ConcurrentModificationException f�hrt.}
\textmd{\\Die in dieser Arbeit bevorzugte Implementation ist, die zu �ndernden Eigenschaften an dem Partikelsystem zu modellieren und �ber Methoden zu �ndern. Erst mit dem n�chsten Aufruf der update-Methode werden die �nderungen anschlie�end auf alle lebendigen Partikel angewendet. Dieser Ansatz kann dann auch ohne Thread-Synchronisation auskommen und f�hrt zu keiner ConcurrentModificationException. Sollte Thread-Synchronisation ben�tigt werden, ist der zu sch�tzende, kritische Bereich mit diesem Ansatz bedeutend kleiner.}

\section{Transparenz zwischen Partikelsystemen}
\textmd{Die Darstellung von Transparenz wurde nur f�r die Partikel eines Partikelsystems untereinander umgesetzt. Die Partikel mehrerer Partikelsysteme untereinander werden nicht sortiert. Um bei mehreren Partikelsystemen alle Partikel mit Transparenz korrekt zeichnen zu k�nnen, m�ssten alle Partikel aller Partikelsysteme wie in Abschnitt \ref{transparency} unabh�ngig von ihrem Partikelsystem back-to-front sortiert werden.}
\textmd{\\Technisch ist das in meinem Aufbau leider nicht so einfach m�glich, da jedes Partikelsystem durch einen eigenen Szenengraphknoten eingebunden und gerendert wird. Der Szenengraphknoten funktioniert dabei autonom und wei� nicht von anderen Szenengraphknoten und ob sie �berhaupt Partikelsysteme darstellen.}

\section{ParticleColorChanger mit anderen Eigenschaften als Lebenszeit}
\textmd{TODO}
	
	%% appendix if used
	%%\appendix
	%%\typeout{===== File: appendix}
	%%\include{appendix}
	
	% bibliography and other stuff
	\backmatter
	
	\typeout{===== Section: literature}
	%% read the documentation for customizing the style
	\bibliographystyle{dinat}
	\bibliography{quellen}
	
	\typeout{===== Section: nomenclature}
	%% uncomment if a TOC entry is needed
	%%\addcontentsline{toc}{chapter}{Glossar}
	\renewcommand{\nomname}{Glossar}
	\clearpage
	\markboth{\nomname}{\nomname} %% see nomencl doc, page 9, section 4.1
	\printnomenclature
	
	%% index
	\typeout{===== Section: index}
	\printindex
	
	\HAWasurency
	
\end{document}
