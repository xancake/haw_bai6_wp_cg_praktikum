\chapter{Einleitung} \label{introduction}
\textmd{Partikelsysteme werden in der Computergraphik zur Simulation von unterschiedlichsten undeutlichen Objekten eingesetzt. Ein Partikelsystem besteht dabei aus einer Vielzahl von einzelnen Partikeln, die zusammen ein Ph�nomen wie zum Beispiel Feuer, Regen, Wolken oder Schnee darstellen.}

\textmd{\\Ihren Ursprung haben Partikelsysteme aus dem Film Star Trek II: The Wrath of Khan. Der Forscher William T. Reeves von Lucasfilm Ltd. entwickelte 1982 das erste Partikelsystem f�r den Genesis Effect, vgl. \cite{natureofcode2012}.}

\section{Aufgabenstellung} \label{problem}
\textmd{Diese Projektarbeit befasst sich mit der Erstellung eines Partikelsystems. Dabei sollen sowohl die Grundlagen wie die Darstellung und Repr�sentation von Partikeln und Partikelsystemen ber�cksichtigt werden, als auch fortgeschrittenere Themen wie Performance und der Umgang mit Transparenz.}

\section{Struktur der Arbeit} \label{structure}
\textmd{Diese Arbeit besteht aus drei Teilen. Im n�chsten Kapitel werden zun�chst Partikelsysteme generell erkl�rt und worauf bei ihrer Implementation geachtet werden muss. Das dritte Kapitel besch�ftigt sich mit der Implementation einer API f�r Partikelsysteme in dem Framework aus dem Wahlpflichtfach Computergrafik. In dem letzten Kapitel werden Erkenntnisse und Probleme aus der Arbeit dargestellt und zuletzt ein Ausblick gegeben, welche weiteren M�glichkeiten Partikelsysteme bieten.}
