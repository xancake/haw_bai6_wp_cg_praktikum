\chapter{Zusammenfassung und Erkenntnisse} \label{summary}
\textmd{In diesem Kapitel werden Erkenntnisse widergespiegelt, die w�hrend der Implementation des Partikelsystems in dem Projekt aufgetreten sind.}


\textmd{
\\ TODOs:
\\
\\ - 
\\ - 
\\ - 
\\ - 
}

\section{Performance in der Update-Methode}
\textmd{In den Update-Methoden sollte m�glichst viel wiederverwendet werden. Insbesondere immutable Datentypen sind hier sch�dlich. Die Update-Methoden werden bei Echtzeitanwendungen mindestens 30 Mal pro Sekunde aufgerufen um einen fl�ssigen Ablauf zu garantieren. Das bietet gro�es Potential f�r ein Speicherleck. In der Update-Methode der Partikel z.B. wurden urspr�nglich bei �nderung der Position und Geschwindigkeit st�ndig neue Vektoren erzeugt. Durch umstellen auf mutable Operationen konnte bei einem Partikelsystem mit 10.000 Partikeln der Speicherbedarf von 660MB auf 330MB reduziert werden.}

\section{Transparenz zwischen Partikelsystemen}
\textmd{Die Darstellung von Transparenz wurde nur f�r die Partikel eines Partikelsystems untereinander umgesetzt. Die Partikel mehrerer Partikelsysteme untereinander werden nicht sortiert. Um bei mehreren Partikelsystemen alle Partikel mit Transparenz korrekt zeichnen zu k�nnen, m�ssten alle Partikel aller Partikelsysteme wie in Abschnitt \ref{transparency} unabh�ngig von ihrem Partikelsystem back-to-front sortiert werden.}
\textmd{\\Technisch ist das in meinem Aufbau leider nicht so einfach m�glich, da jedes Partikelsystem durch einen eigenen Szenengraphknoten eingebunden und gerendert wird. Der Szenengraphknoten funktioniert dabei autonom und wei� nicht von anderen Szenengraphknoten und ob sie �berhaupt Partikelsysteme darstellen.}