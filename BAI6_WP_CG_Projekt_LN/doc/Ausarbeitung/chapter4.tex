\chapter{Zusammenfassung und Erkenntnisse} \label{summary}
\textmd{In diesem Kapitel werden Erkenntnisse widergespiegelt, die w�hrend der Implementation des Partikelsystems in dem Projekt aufgetreten sind.}

\section{Performance in der Update-Methode}
\textmd{Die Update-Methoden von Partikeln und Partikelsystemen werden mindestens 30 mal pro Sekunde aufgerufen um einen fl�ssigen Bildlauf zu garantieren. Dazu kommt, dass es pro Partikelsystem tausende von Partikeln geben kann. Deshalb sollten in den Update-Methoden so wenig Objekte wie m�glich erzeugt werden. Insbesondere immutable Operationen bzw. Datentypen sind hier sch�dlich, da sie neue Objekte erzeugen. Das bietet gro�es Potential f�r ein Speicherleck. In der Update-Methode der Partikel z.B. wurden urspr�nglich bei �nderung der Position und Geschwindigkeit st�ndig neue Vektoren erzeugt. Durch umstellen auf mutable Operationen konnte bei einem Partikelsystem mit 10.000 Partikeln der Speicherbedarf von 660MB auf 330MB reduziert werden.}

\section{Eigenschaften von Partikeln an Partikelsystemen �ndern}
\textmd{Sollen Eigenschaften von allen Partikeln eines Partikelsystems auf einmal ge�ndert werden, wie das zum Beispiel beim Anwenden von Kr�ften der Fall ist, muss darauf R�cksicht genommen, dass die update-Methode ggf. von einem anderen Thread ausgef�hrt wird. Bei einer Implementation die z.B. in der applyForce-Methode den Partikeln direkt durch eine Schleife eine Kraft zuweisen m�chte kann es passieren, dass sich die Sammlung der lebendigen Partikel konkurrierend �ndert, was zu einer ConcurrentModificationException f�hrt.}
\textmd{\\Die in dieser Arbeit bevorzugte Implementation ist, die zu �ndernden Eigenschaften an dem Partikelsystem zu modellieren und �ber Methoden zu �ndern. Erst mit dem n�chsten Aufruf der update-Methode werden die �nderungen anschlie�end auf alle lebendigen Partikel angewendet. Dieser Ansatz kann dann auch ohne Thread-Synchronisation auskommen und f�hrt zu keiner ConcurrentModificationException. Sollte Thread-Synchronisation ben�tigt werden, ist der zu sch�tzende, kritische Bereich mit diesem Ansatz bedeutend kleiner.}

\section{Transparenz zwischen Partikelsystemen}
\textmd{Die Darstellung von Transparenz wurde nur f�r die Partikel eines Partikelsystems untereinander umgesetzt. Die Partikel mehrerer Partikelsysteme untereinander werden nicht sortiert. Um bei mehreren Partikelsystemen alle Partikel mit Transparenz korrekt zeichnen zu k�nnen, m�ssten alle Partikel aller Partikelsysteme wie in Abschnitt \ref{transparency} unabh�ngig von ihrem Partikelsystem back-to-front sortiert werden.}
\textmd{\\Technisch ist das in meinem Aufbau leider nicht so einfach m�glich, da jedes Partikelsystem durch einen eigenen Szenengraphknoten eingebunden und gerendert wird. Der Szenengraphknoten funktioniert dabei autonom und wei� nicht von anderen Szenengraphknoten und ob sie �berhaupt Partikelsysteme darstellen.}

\section{ParticleColorChanger mit anderen Eigenschaften als Lebenszeit}
\textmd{TODO}